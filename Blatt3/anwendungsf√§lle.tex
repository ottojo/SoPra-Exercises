\documentclass[a4paper]{article}

\usepackage[utf8]{inputenc}
\usepackage[ngerman]{babel}
\usepackage{tikz}
\usepackage{tikz-uml}
\usetikzlibrary{positioning}


\author{Jonas Otto}
\title{SoPra Übungsblatt 3}

\begin{document}
\maketitle

\section{UML-Anwendungsfalldiagramm}
\begin{tikzpicture} 
    \umlactor[y=-1]{Spieler}
    \umlactor[y=-6]{Nutzer}
    \begin{umlsystem}[x=4, y=0]{HEXXAGON} 
        \umlusecase[x=3, y=-1, name=spielen]{Spielen}
        \umlusecase[name=bewegen, width=2cm]{Spielstein bewegen}
        \umlusecase[y=-2, name=verdoppeln, width=2cm]{Spielstein verdoppeln}
        \umlusecase[y=-4, width=2cm, name=lerstellen]{Lobby erstellen}
        \umlusecase[y=-6, width=2cm, name=lverlassen]{Lobby verlassen}
        \umlusecase[y=-8, width=2cm, name=lbeitreten]{Lobby beitreten}
        \umlusecase[y=-7, x=3, width=2cm, name=lnav]{Lobby navigieren}
    \end{umlsystem} 
    \umlinherit{bewegen}{spielen}
    \umlinherit{verdoppeln}{spielen}
    \umlassoc{Spieler}{bewegen}
    \umlassoc{Spieler}{verdoppeln}
    \umlinherit{lverlassen}{lnav}
    \umlinherit{lbeitreten}{lnav}
    \umlassoc{Nutzer}{lerstellen}
    \umlassoc{Nutzer}{lverlassen}
    \umlassoc{Nutzer}{lbeitreten}

\end{tikzpicture}

\section{UML-Zustandsdiagramm}
\subsection{Client}
\begin{tikzpicture}
    \umlbasicstate[x=0, y=0, name=menu]{Hauptmenü}
    \umlbasicstate[below=2cm of menu, name=lobbyoverview]{Lobbyübersicht}
    \umlbasicstate[below=2cm of lobbyoverview, name=lobby]{Lobby}
    \begin{umlstate}[x=6, name=game]{Spiel}
        \umlbasicstate[x=0, y=0, name=zug]{Am Zug}
        \umlbasicstate[x=0, y=-3, name=gegnerZug]{Gegner am Zug}
        \umltrans[arg=srv.Move, pos=0.5]{zug}{gegnerZug}
        \umlHVHtrans[arm1=3cm, arg=srv.Move, pos=0.5]{gegnerZug}{zug}
    \end{umlstate}

    \umlstateinitial[name=initial, y=3]
    \umltrans[arg=Starten, pos=0.5]{initial}{menu} 

    \umltrans[arg=Server wählen, recursive=-20|10|3.5cm, pos=1.5, recursive direction=right to right]{menu}{menu}
    \umltrans{menu}{lobbyoverview}
    \umltrans[arg=Betreten, pos=0.5]{lobbyoverview}{lobby}
    \umlHVHtrans[arg=Verlassen,pos=0.8, arm1=-2cm, anchor2=190]{lobby}{lobbyoverview}
    \umlHVtrans[arg=Starten, pos=0.5]{lobby}{game}
    \umlHVHtrans[arg=Verlassen,pos=0.8, arm1=-2cm, anchor1=170]{lobbyoverview}{menu}
    \umlHVtrans[arg=Spielende, pos=0.5, anchor2=290]{game}{menu}
\end{tikzpicture}

\subsection{Server}
\begin{tikzpicture}
    \umlstateinitial[name=initial, y=3, x=4]
    \begin{umlstate}[name=game, x=0, y=0]{Spiel}
        \umlbasicstate[y=0, name=p1]{Waiting P1}
        \umlbasicstate[y=-3, name=p2]{Waiting P2}
        \umltrans[anchor1=260, anchor2=100]{p1}{p2}
        \umltrans[anchor1=80, anchor2=280]{p2}{p1}
    \end{umlstate}

    \begin{umlstate}[name=lobby, x=4, y=0.5]{Lobby}
    \end{umlstate}

    \umlstatefinal[name=final, below=1cm of game]

    \umltrans{initial}{lobby}
    \umlVHtrans[arg={srv.Start /c*.Start}, pos=1.5, text width=2cm]{lobby}{game}
    \umltrans[arg={srv.End /c*.End}, pos=0.5, text width=2cm]{game}{final}
\end{tikzpicture}

\section{UML-Sequenzdiagramm}
\begin{tikzpicture}
    \begin{umlseqdiag}
        \umlactor[class=Player]{a}
        \umlactor[class=Player, x=8]{b}

        \umlcreatecall[class=Lobby, x=4]{a}{l}
        \begin{umlcall}[op={join}, dt=7]{a}{l}
        \end{umlcall}
        \begin{umlcall}[op={join}, dt=8]{b}{l}
        \end{umlcall}
        \umlcreatecall[class=Game, x=4]{a}{g}

        \begin{umlfragment}[type=game, inner xsep=2] 

            \begin{umlcall}[op={move(i,j)}, return=success, dt=7]{a}{g}
            \end{umlcall}

            \begin{umlcall}[op={move(i,j)}, return=fail, dt=28]{b}{g}
            \end{umlcall}
            \begin{umlcall}[op={move(i,j)}, return=success, dt=6]{b}{g}
            \end{umlcall}

            \begin{umlcall}[op={move(i,j)}, return=not allowed, dt=6]{b}{g}
            \end{umlcall}

            \begin{umlcall}[op={move(i,j)}, return=success, dt=33]{a}{g}
            \end{umlcall}
        \end{umlfragment}

        \begin{umlcall}[op=end(lost), dt=7]{g}{a}
            \begin{umlcall}[op=end(won)]{g}{b}
                \end{umlcall}
        \end{umlcall}

    \end{umlseqdiag}
\end{tikzpicture}

\end{document}
