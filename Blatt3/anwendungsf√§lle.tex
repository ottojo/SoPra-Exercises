\documentclass[a4paper]{article}

\usepackage[utf8]{inputenc}
\usepackage[ngerman]{babel}
\usepackage{tikz}
\usepackage{tikz-uml}


\author{Jonas Otto}
\title{SoPra Übungsblatt 3}

\begin{document}
\maketitle

\section{UML-Anwendungsfalldiagramm}
\begin{tikzpicture} 
    %\draw[style=help lines] (0,0) grid  (3,3);
    \umlactor[y=-1]{Spieler}
    \umlactor[y=-6]{Nutzer}
    \begin{umlsystem}[x=4, y=0]{HEXXAGON} 
        \umlusecase[x=3, y=-1, name=spielen]{Spielen}
        \umlusecase[name=bewegen, width=2cm]{Spielstein bewegen}
        \umlusecase[y=-2, name=verdoppeln, width=2cm]{Spielstein verdoppeln}
        \umlusecase[y=-4, width=2cm, name=lerstellen]{Lobby erstellen}
        \umlusecase[y=-6, width=2cm, name=lverlassen]{Lobby verlassen}
        \umlusecase[y=-8, width=2cm, name=lbeitreten]{Lobby beitreten}
        \umlusecase[y=-7, x=3, width=2cm, name=lnav]{Lobby navigieren}
    \end{umlsystem} 
    \umlinherit{bewegen}{spielen}
    \umlinherit{verdoppeln}{spielen}
    \umlassoc{Spieler}{bewegen}
    \umlassoc{Spieler}{verdoppeln}
    \umlinherit{lverlassen}{lnav}
    \umlinherit{lbeitreten}{lnav}
    \umlassoc{Nutzer}{lerstellen}
    \umlassoc{Nutzer}{lverlassen}
    \umlassoc{Nutzer}{lbeitreten}

\end{tikzpicture}

\end{document}
